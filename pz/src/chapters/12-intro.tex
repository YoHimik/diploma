\Introduction

Веб технологии широко распространились в нашем мире и на сегодняшний день почти каждая ВС взаимодействует со всемирной паутиной.
В свою очередь, поддержка и разработка наиболее популярной архитектуры <<Клиент-Сервер>> таких систем требует существенных временных затрат, поскольку уже с самого начала проектирования требуется решить ряд следующих задач:

\begin{itemize}
    \item вертикальное и горизонтальное масштабирование системы,
    \item доставка обновлений сервиса на рабочие ЭВМ,
    \item управление окружением ВС,
    \item осуществления контроля качества поступающих изменений,
    \item управление версиями ВС,
    \item бесшовное развёртывание отдельных компонентов системы,
    \item оперативная загрузка срочных исправлений.
\end{itemize}

Для решения данных задач в современном мире применяется DevOps методология разработки ПО\cite{projectPhoenix}.
Основная идея заключается в предоставлении удобных инструментов связи разработчиков, системных администраторов и не только.

DevOps включает в себя множество различных технологий, среди которых можно выделить
SaaS --- это облачное решение при использовании \cite{cd}
которого, пользователь получает доступ к сервису, как правило, через браузер или по API;
IaC --- это процесс управления и провизионирования датацентров и серверов с помощью машиночитаемых файлов определений,
созданный как альтернатива физическому конфигурированию оборудования и оперируемым человеком инструментам;
а так же CI/CD --- это методология обеспечения оперативности вывода новой функциональности продукта и повышение качества разрабатываемого решения\cite{ciCd}.

Теперь, вместо того, чтобы запускать сотню различных файлов конфигурации,
достаточно запускать скрипт, который управляет инфраструктурой и масштабированием системы.

Несмотря на это, для ввода в рабочий процесс всё равно необходима дополнительная разработка механизмов развёртки,
начиная от описания файлов конфигураций до разработки дополнительных порграммных средств.

В зависимости от решаемых задач и условий эксплуатации разработчиками выбираются конкретные DevOps инструменты.

Разрабатываемая в рамках данной работы система автоматизации управления жизненным циклом веб-сервиса,
согласно техническому заданию, должна отвечать следующим требованиям:

\begin{itemize}
    \item использовать современные DevOps методологии,
    \item взаимодействие с конечным пользователем по методолгии SaaS,
    \item возможность добавления в систему не описанных заранее сервисов по методолгии IaC,
    \item использование методологии CI/CD для взаимодействия с инфраструктурой и организации контроля качества,
    \item наличие конфигурации прав доступа (приватизации исходного кода),
    \item наличие хранилища пакетов, образов и контейнеров,
    \item стоимость установки и обслуживания системы --- не более 12 500 рублей за установку и не более 5 000 рублей в месяц за обслуживание на момент написания данной работы.
\end{itemize}

Поставленные требования определяют предмет исследования:

Простое и доступное в установке и поддержке ПО для автоматизации управления жизненным циклом веб-сервиса\cite{vkrsen}.

Объект исследования: ПО, задействованные и которые могут быть использованы для автоматизации управления жизненным циклом веб-сервиса.

При проведении исследований были просмотрены 20 источников, в том числе 11 электронных ресурсов, что отражено в библиографическом списке.
При написании выпускной квалификационной работы (ВКР) из библиографического списка использовано 5 источников.

ВКР состоит из введения, основной части, включающей четыре главы, заключения и приложений.
В ВКР содержится 31 страница основного текста, 11 рисунков, 3 таблицы и 0 приложений.