\Introduction

В современном мире информационных технологий крайне распространена DevOps методология разработки ПО\cite{projectPhoenix}.
Тем не менее развёртка рабочего окружения занимает значительное время, поскольку включает множество задач от
описания файлов конфигураций до разработки дополнительных программных средств развёртывания.

В практических целях автору был предложен веб-сервис для развёртывания.
Результаты выполнения задания должны быть представлены в выпускной квалификационной работе.

На основании выданного технического задания, определяется цель исследования: развернуть и автоматизировать управление жизненным циклом веб-сервиса на базе Node.js.

Для реализации цели исследования ставятся следующие задачи исследования:

\begin{itemize}
    \item провести обзор необходимых средств для автоматизации управления жизненным циклом веб-сервиса,
    \item рассмотреть полученный для развёртывания веб-сервис и проанализировать требования,
    \item провести проектирование механизмов автоматизации управления жизненного цикла веб-сервиса,
    \item составить план тестирования механизмов развёртывания веб-сервиса,
    \item провести практические работы по развёртке и автоматизации управления жизненного цикла веб-сервиса,
    \item провести тестирование механизмов развёртывания и обосновать полученные результаты.
\end{itemize}

Поставленные задачи определяют предмет исследования:

Основные параметры имеющихся DevOps решений с целью определения наиболее подходящих из них для развёртывания веб-сервисов
по наиболее популярным и современным методологиям IaC, CI/CD и SaaS.

Объект исследования: программные и аппаратные средства, задействованные и которые могут быть использованы для автоматизации управления жизненным циклом веб-сервиса.

При проведении исследований были просмотрены 20 источников, в том числе 11 электронных ресурсов, что отражено в библиографическом списке.
При написании выпускной квалификационной работы (ВКР) из библиографического списка использовано 5 источников.

ВКР состоит из введения, основной части, включающей четыре главы, заключения и приложений.
В ВКР содержится 42 страница основного текста, 16 рисунков, 4 таблицы, 5 листингов и 0 приложений\cite{vkrsen}.
