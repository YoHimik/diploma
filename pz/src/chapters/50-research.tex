\chapter{ТЕСТИРОВАНИЕ СИСТЕМЫ АВТОМАТИЗАЦИИ УПРАВЛЕНИЯ ЖИЗНЕННЫМ ЦИКЛОМ ВЕБ-СЕРВИСА}
\label{cha:research}

\section{Проведение ручного тестирования}

Описание процесса тестирования, машина, скриншоты, непредвиденные ситуации и решение \cite{devOpsPhy}

\section{Анализ результатов тестирования}

Результаты тестирования представлены в виде таблицы:

\begin{center}
    \begin{longtable}{|p{0.15\textwidth}|p{0.2\textwidth}|p{0.2\textwidth}|p{0.2\textwidth}|}
        \caption{Результаты тестирования}
        \label{tab:testing-res}
        \hline
        Название теста & Фактический результат & Ожидаемый результат & Комментарий \\
        \hline
        Релиз новой версии в develop окружении & Релиз успешно создаётся & Релиз успешно создаётся & - \\
        \hline
        Релиз новой версии в testing окружении & Релиз успешно создаётся & Релиз успешно создаётся & - \\
        \hline
        Релиз новой версии в release окружении & Релиз успешно создаётся & Релиз успешно создаётся & Время работы увеличино \\
        \hline
    \end{longtable}
\end{center}

Все тесты пройдены успешно согласно плану, несмотря на небольшие проблемы с производительностью.


%%% Local Variables:
%%% mode: latex
%%% TeX-master: "rpz"
%%% End:
