\Conclusion % заключение к отчёту

Результаты выполнения задания на выпускную квалификационную работу, - позволяют сделать следующие выводы:
\begin{itemize}
    \item
    Цель исследования: развернуть и автоматизировать управление жизненным циклом веб-сервиса на базе Node.js, - полностью реализована;
    вариант предложенного решения развёртывания позволяет быстро ввести веб-сервис на базе Node.js в рабочее состояние и автоматизировать управление его жизненным циклом по современным методологиям DevOps.

    \item
    В рамках решения задачи: Провести обзор необходимых средств для автоматизации управления жизненным циклом веб-сервиса,
    - были рассмотрены преимущества системы контроля версий Git, проанализированы основные технические параметры и отличия между различными Git хостингами, а так же инструментами оркестрации контейнеров Docker Swarm и Kubernetes .

    \item
    В рамках решения задачи: Рассмотреть полученный для развёртывания веб-сервис и проанализировать требования,
    - была составлена и описана диаграмма развёртывания веб-сервиса, была установлена необходимость хранилища пакетов и образов, были описаны актёры и случаи использования системы в виде соответствующей диаграммы.

    Отдельно были описаны рабочие окружения веб-сервиса и на основании необходимых для развёртывания компонентах были составлены виды репозиториев в системе.

    \item
    В рамках решения задачи: Провести проектирование механизмов автоматизации управления жизненного цикла веб-сервиса,
    - были выбраны и автоматизированы два основных вида тестирования веб-сервиса с предоставлением аналитических данных сотрудникам отдела качества,
    была составлена и задокументирована диаграмма компонентов конфигураций внутри репозиториев системы.

    Отдельное внимание было уделено применению модели ветвления Git Flow вместе с инструментами автоматизации CI/CD в целях автоматизированного создания релизов сервисов и библиотек,
    а так же нахождению компонентов для обновления при помощи обхода графа зависимостей.

    Так же была спроектирована диаграмма компонентов репозиториев для организации хранения исходного кода отдельно от конфигурационных файлов развёртки веб-сервиса.

    \item
    В рамках решения задачи: Составить план тестирования механизмов развёртывания веб-сервиса,
    - были сформулированы цели и задачи тестирования, была выбрана методология тестирования, а так же были описаны тестовые сценарии.

    \item
    В рамках решения задачи: Провести практические работы по развёртке и автоматизации управления жизненного цикла веб-сервиса,
    - в GitLab была создана группа проекта, были созданы необходимые репозитории согласно диаграммам и загружен исходный код веб-сервиса.

    На рабочем сервере были открыты TCP и UDP порты, а так же были введены требуемые команды для инициализации кластера Docker Swarm.
    Отдельно был зарегистрирован GitLab Runner и сконфигурирован под оптимальную работу с кешированием в системе.

    \item
    В рамках решения задачи: Провести тестирование механизмов развёртывания и обосновать полученные результаты,
    - было произведено ручное тестирование по методологии <<Белого ящика>> функций полученной системы.
\end{itemize}


%%% Local Variables:
%%% mode: latex
%%% TeX-master: "rpz"
%%% End:
