\section{ОБЗОР И АНАЛИЗ СУЩЕСТВУЮЩИХ СИСТЕМ}

С 28.11.2021 по 02.12.2021 выполнялся этап обзора и анализа существующих аналогичных программных комплексов.

Для корректного формулирования требований к разрабатываемому комплексу автоматизации управления жизненным циклом веб-сервиса,
необходимо проанализировать и сравнить характеристики схожих по назначению решений, применяемых в данный момент на архитектуре <<Клиент-Сервер>>,
либо по своим характеристикам подходящих для такого применения.

Рассматривая рынок развёртки программного обеспечения, стоит начать с такого ключевого элемента, как системы контроля версий (или VCS --- Version Control System).
Так как в мире современных технологий большая часть практических задач решается именно такими системами.

Наиболее популярной системой контроля версий на момент написания работы является git.
Git --- система управления версиями программного обеспечения с распределенной архитектурой.
В отличие от некогда популярных систем вроде CVS и Subversion (SVN),
где полная история версий проекта доступна лишь в одном месте,
в Git каждая рабочая копия кода сама по себе является репозиторием.
Это позволяет всем разработчикам хранить историю изменений в полном объеме.
Разработка в Git ориентирована на обеспечение высокой производительности, безопасности и гибкости распределенной системы.

Сам по себе git предоставляет только инструменты для локальной разработки и имеет весьма ограниченный функционал.
Для работы в большинстве случаев выбирается SaaS Git хостинг исходного кода.
Такие сервисы предоставляют удобный масштабируемый хостинг для Git-репозиториев с веб-интерфейсом для просмотра и редактирования кода,
а также гибкими настройками доступа.
Современные и простые способы организации процессов CI/CD и решения самых разных задач с их помощью.

В настоящее время на рынке имеется несколько лидерующих представителей, отличающихся в основном стоимостью и набором дополнительных инструментов.

Наиболее популярным из таких хостингов является GitHub от международной компании Microsoft.
Для проектов с открытым исходным кодом сервис бесплатен.
Функции приватизации и контроля доступа для команд предоставляются за отдельную плату.
Для развёртки проектов по методологии CI/CD, GitHub предоставляет инструмент Actions.
С его помощью автоматизация процессов развёртки и тестирования производится в декларативном стиле путём описания YAML файлов.
Хранилище образов и пакетов предоставляется с ограничением и только проектам с открытм исходным кодом.
В целом, данный хостинг лучше всего подходит для таких проектов и редко используется командами разработчиков в коммерческой сфере деятельности.

Следующим из таких хостингов является Bitbucket от австралийской компании Atlassian.
Данный хостинг более акцентирован на коммерческую разработку и предоставляет функции приватизации бесплатно для небольших команд.
Возможности CI/CD аналогичны GitHub Actions, только предоставляются продуктом Bamboo.
Хранилище образов и пакетов отсутствует.
Bitbucket следует рассматривать, как альтернативу GitHub только для коммерческой разработки.

Одним из наиболее подходящих хостингов является GitLab от украинских разработчик (ныне зарубежной компании GitLab Inc).
Ключевой особенностью является то, что этот сервис изначально разрабтывался,
как полноценная система для управления жизненным циклом программного обеспечения на всех этапах разработки.
Благодаря этому в нём реализовано большинство основных функций для обеспечения полноценного рабочего окружения по методологии CI/CD.
Бесплатно сервис предоставляет, как приватизацию и управление доступом к исходному коду, так и хранилища образов и пакетов.
Возможности непрерывной интеграции и доставки включают в себя отчеты о тестировании в реальном времени, параллельное выполнение, локальный запуск скриптов, поддержку Docker.
Несмотря на это, для ввода его в рабочий процесс всё равно необходима дополнительная разработка механизмов развёртки,
начиная от описания YAML файлов конфигураций до разработки дополнительных порграммных средств.

Так же был рассмотрен продукт от разработчиков из Санкт-Петербурга (международной компании JetBrains) --- Space.
Данное решение является интегрированной средой для командной работы, которая включает управление исходным кодом, постановку и работу над задачами, управление командами разработчиков и инструменты коммуникации.
Все функции Space так же предоставляет бесплатно, но с ограничениями в разумных пределах, подходящими небольшим командам разработчиков.
Несмотря на большое количество возможностей, данный продукт существует на рынке сравнительно небольшое количество времени и большинство ключевых функций ещё не реализовано.

Помимо git хостинга данному проекту так же потребуется механизм масштабирования системы.
Технология контейнеризации (Docker) позволяет запускать приложения в отдельных независимых средах --- контейнерах.
Они упрощают развертывание приложений, изолируют их друг от друга и ускоряют разработку.
Но когда контейнеров становится слишком много, ими трудно управлять.
Тут на помощь приходят IaC системы оркестрации.

У Docker есть стандартный инструмент оркестрации --- Docker Swarm.
Он поставляется вместе с Docker, довольно прост в настройке и позволяет создать кластер в кратчайшие сроки.
Из минусов следует отметить узкую функциональность: возможности ограничены Docker API.
Это значит, что Swarm способен сделать лишь то, что позволяют возможности Docker.

Альтернативным решением является Kubernetes --- это универсальное средство для создания распределенных систем.
Это комплексная система с большим количеством возможностей, разработка и поддержка которой самостоятельно значительно трудозатратна.
Kubernetes мощный инструмент, имеющий много возможностей, которые позволяют строить действительно комплексные распределенные системы.
К минусам относится управление, как оно использует отдельный набор команд и инструментов, несовместимых с Docker CLI.
Гибкость данного решения излешне и лишь затруднит введение проекта в рабочее состояние в контексте данной работы.

Подробное сравнение git хостингов и оркестраторов представлено в тексте ВКР в виде таблиц.

Подводя итоги исследования доступных решений систем автоматизации управления жизненным циклом веб-сервисов, можно сделать следующие выводы:
\begin{itemize}
    \item анализ источников показывает, что на рынке отсутствуют подходящие под требования проектируемой системы готовые решения,
    позволяющие осуществлять дальнейшие доработки системы под конкретные задачи;
    \item анализ использованных источников показывает, что цели и задачи работы возможно реализовать на основе Git хостинга GitLab,
    разработав программные механизны части системы.
    Правильный выбор сервисных компонентов системы позволит выполнить требования технического задания и достигнуть цели работы.
\end{itemize}



