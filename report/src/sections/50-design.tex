\section{ПРОЕКТИРОВАНИЕ СИСТЕМЫ}

Этап проектирования происходил в период с 03.12.2021 по 07.12.2021.

Проводя обзор доступных на рынке git хостингов, можно сделать вывод, что наиболее распространенным git хостингом на сегодняшний день является хостинг компании GitLab Inc.
К тому же, по соотношению цена-функционал хостинг этой компании существенно обходит конкурентов.
Также, к существенному преимуществу можно отнести наличие обширного сообщества пользователей и разработчиков программных решений на основе git хостинга GitLab,
что позволяет иметь доступ к множеству готовых решений и получать помощь в разработке при необходимости.

Для реализации поставленной в данной работе задачи гибкость настройки всей инфраструктуры окружения не требуется, а также ставится в приоритет скорость ввода в рабочее состояние.
Поэтому в качестве оркестратора вместо гибкости Kubernetes был выбран Docker Swarm.
Но так как в данной работе делается акцент на гибкость всей системы, то далее будет рассмотрена описание конфигурации для использования с Kubernetes, не считая уставноку и настройку самого кластера.
Для работы будет подготовлена одна вершина Docker Swarm в статусе менеджер, поскольку более не требуется на данном этапе.

Как было сказано ранее, для развёртки в работе используется Docker, поэтому необходим простой инструмент удаленного доступа к сокету Docker сервиса на рабочем сервере.
Для этих целей будет использоваться GitLab Runner с установленным исполнителем задач Docker.
Кратко описать работу GitLab Runner можно следующим образом: GitLab Runner запускается в отдельном контейнере с добавленным volume на сокет Docker,
таким образом получается избежать достаточно сложного и нецелесообразного запуска Docker внутри Docker,
так как в этом случае GitLab Runner получает доступ напрямую к сокету Docker сервера.
Данное решение имеет потенциальную проблему с безопасностью, поскольку если злоумышленник получит доступ к описанию задач GitLab CI/CD, то он сможет запускать на рабочем сервере любое ПО.
Для избежания данной проблемы будут установлены настройки доступа внутри GitLab.
Так же для избежания потери полезного времени работы GitLab Runner, необходимо будет произвести настройку кэша GitLab Runner.
Ключевой настройкой является политика загрузки образов для задач, поскольку по умолчанию GitLab Runner в любом случае будет загружать образ из регистра, даже если образ представлен локально.
Согласно требованиям GitLab Runner должен будет запускать минимум три задачи за единицу времени, данное значение будет отражено в конфигурации на этапе реализации.

Автоматизация контроля качества будет реализована путём запуска описанных разработчиками тестовых сценариев внутри Pipelines.
Отчёты по прохождению сценариев будут представлены в веб интерфейсе GitLab путём интеграции с CI/CD.

Для контроля версий будет использована модель ветвления git flow со следующими окружениями:

\begin{itemize}
    \item develop --- инсценировка рабочего окружения веб сервиса для разработчиков,
    \item testing --- окружение для проведения ручного тестирования,
    \item release --- рабочее окружение сервиса для реальных пользователей.
\end{itemize}

Основная проблема заключается в обновлении пакетов зависимостей проекта, поскольку для обновления необходимо пройти по зависимым сервисам и построить граф.
Для решения этой проблемы будет разработано программное решение на базе CI/CD c применением bash скриптов при помощи package.json файлов веб сервиса.

Для проведения тестирования был составлен план содержащий набор ключевых функций для тестирования.
Подробнее план представлен в виде таблицы с описанием сценариев, входных данных и ожидаемого результата в тексте работы ВКР.

