\newpage

Актуальность темы.

Веб технологии широко распространились в нашем мире и на сегодняшний день почти каждая ВС взаимодействует со всемирной паутиной.
В свою очередь, поддержка и разработка наиболее популярной архитектуры <<Клиент-Сервер>> таких систем требует существенных временных затрат, поскольку уже с самого начала проектирования требуется решить ряд следующих задач:

\begin{itemize}
    \item вертикальное и горизонтальное масштабирование системы,
    \item доставка обновлений сервиса на рабочие ЭВМ,
    \item управление окружением ВС,
    \item осуществления контроля качества поступающих изменений,
    \item управление версиями ВС,
    \item бесшовное развёртывание отдельных компонентов системы,
    \item оперативная загрузка срочных исправлений.
\end{itemize}

Подавляющее большинство этих задач решается при помощи методологии SaaS (Software as a Service, <<Программное обеспечение как услуга>>).
Основная идея заключается в предоставлении программного обеспечения по подписке.
SaaS --- это облачное решение при использовании которого, пользователь получает доступ к сервису, как правило, через браузер или по API.
Оплачивает доступ и максимально быстро получает на руки готовый инструмент.

В данной работе сделан акцент на применение готовых SaaS решений, поскольку в условиях современной коммерческой разработки игнорирование таких
технологий по большей части нецелесообразно и влечёт серьёзные временные, и как следствие, денежные потери.

Помимо SaaS так же используется методология IaC (Infrastructure as Code, <<Инфраструктура как код>>) --- это процесс управления и позиционирования дата центров и серверов с помощью машиночитаемых файлов определений,
созданный как альтернатива физическому конфигурированию оборудования и оперируемым человеком инструментам.
Теперь, вместо того, чтобы запускать сотню различных файлов конфигурации,
IaC позволяет просто запускать скрипт, который управляет инфраструктурой и масштабированием системы.

Большинство таких средств использует в качестве основы Docker, который предоставляет инструментарий для контейнеризации программного обеспечения.

Так как в полученном для развёртки сервисе уже используется Node.js и Docker, то для корректной работы разрабатываемого комплекса требуется наличие регистров Node пакетов и Docker образов.

Так же в наши дни крайне распространена методология разработки CI/CD (Continuous Integration/Continuous Deployment, <<Продолжительная интеграция/Продолжительная развёртка>>).
Она представляет собой автоматизацию тестирования и доставки новых проектов разрабатываемого проекта разработчикам, аналитикам, инженерам качества, конечным пользователям и другим заинтересованным сторонам.
Метод обеспечивает оперативность вывода новой функциональности продукта и повышение качества разрабатываемого решения.
Методология предоставляет интерфейс конвейеров (Pipelines), состоящий из последовательно выполняемых задач (Jobs).

В зависимости от решаемых задач и условий эксплуатации разработчиками выбираются конкретные инструменты из области рассмотренных выше методологий.

Разрабатываемая в рамках данной работы система автоматизации управления жизненным циклом веб-сервиса,
согласно техническому заданию, должна отвечать следующим требованиям:

\begin{itemize}
    \item взаимодействие с конечным пользователем по методологии SaaS,
    \item возможность добавления в систему не описанных заранее сервисов по методологии IaC,
    \item использование методологии CI/CD для взаимодействия с инфраструктурой и организации контроля качества,
    \item наличие конфигурации прав доступа (приватизации исходного кода),
    \item наличие хранилища NPM пакетов и регистра Docker образов,
    \item стоимость установки и обслуживания системы --- не более 12 500 рублей за установку и не более 5 000 рублей в месяц за обслуживание на момент написания данной работы.
\end{itemize}
