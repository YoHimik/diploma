\section{ОБЗОР ЛИТЕРАТУРЫ ПО ТЕМЕ ВКР}

Данный этап практики выполнялся с 23.11.2021 по 27.11.2021.
Автор отчета ознакомился с требованиями к выпускным квалификационным работам (утвержденным УС ИТМО 24.03.2020), подобрал литературу (книги, научные статьи, электронные ресурсы) не старше 5 лет по теме ВКР.

Для реализации цели работы, описанной во введении, были поставлены следующие задачи:
\begin{enumerate}
    \item рассмотреть полученный для разёртки веб-сервис,
    \item рассмотреть имеющиеся на рынке системы по автоматизации управления жизенным циклом веб-сервисов,
    \item разработать принцип построения сервисной части комплекса,
    \item выбрать компоненты для сервисной части комплекса,
    \item произвести настройку составляющих элементов сервисной части комлпекса,
    \item разработать программную часть комплекса,
    \item собрать опытный стенд, провести тестирование опытного образца комплекса.
\end{enumerate}

Согласно поставленным задачам, предметами исследований в рамках работы являются параметры доступных технологических
решений с последующим выбором наиболее подходящих для проекта, а также параметры доступных электронных и программных
компонентов в рамках выбранного решения.

Объектами исследования являются системы автоматизации развёртки, представленные на рынке на момент выполнения данной работы, а также другие доступные для реализации программные средства.
В процессе выполнения работы были просмотрены 22 источника, в том числе 12 электронных ресурсов, которые перечислены в соответствующем разделе отчета.
